\documentclass[11pt]{article}

%%
%% PACKAGES
%%

% Margins
\usepackage[margin=0.9in, top=0.8in, bottom=1.0in]{geometry}

% Fonts, typesetting, and math symbols
\usepackage[T1]{fontenc}
\usepackage{tgpagella} % Palatino-based font from TeX Gyre
\usepackage[scaled]{beramono} % Lovely monospace font
\usepackage{tgheros} % Helvetica-based font for headings
\usepackage{amsmath, amssymb}
\usepackage{mathtools}
\usepackage{mathdots}
\usepackage{microtype}
\usepackage{xspace}
\usepackage{xfrac}
\usepackage{calc}

% Plotting and drawing
\usepackage{tikz} % This automatically loads graphicx!
%\usetikzlibrary{calc} % For relative positions to defined coords
%\usepackage{pgfplots} % Scientific plotting tools
%\pgfplotsset{compat=1.7}

% Graphics
%\usepackage{graphicx}
%\usepackage[update,prepend]{epstopdf}
\usepackage{titlesec}
\usepackage{color}

% Table improvements
\usepackage{booktabs}

% Figure placement
%\usepackage{wrapfig}

% Code listings
\usepackage{listings}
\usepackage{matlab-prettifier}	% MATLAB code listings

% Tweaks for captions and enumerations
\usepackage[labelfont=bf]{caption}
%\captionsetup[wrapfigure]{margin=0.5cm}
\usepackage{enumitem}
\setlist[itemize]{itemsep=3pt,leftmargin=*,label=\textbullet}

% Fancy headers
\usepackage{fancyhdr}
\setlength{\headheight}{0pt}
\setlength{\footskip}{50pt}
\renewcommand{\headrulewidth}{0pt}
\renewcommand{\footrulewidth}{0pt}

%%
%% SETTINGS
%%

% Path to look for graphics
\graphicspath{{../images/}}

% Caption spacing
\setlength{\abovecaptionskip}{0pt}

% List spacing
\setlist{noitemsep}

% Table spacing
\renewcommand{\arraystretch}{1.3}

% Math operator font
% Note that cmr=roman and cmss=sans-serif.
\DeclareSymbolFont{sfoperators}{OT1}{cmr}{m}{n}
\DeclareSymbolFontAlphabet{\mathsf}{sfoperators}
\makeatletter
\def\operator@font{\mathgroup\symsfoperators}
\makeatother

%% No indent all paragraphs
%\setlength{\parindent}{0in}

% Figure and table references
\newcommand{\figref}[1]{Figure~\ref{#1}}
\newcommand{\tabref}[1]{Table~\ref{#1}}

% Special format section headings
\titleformat{\section}%
	{\color{blue}\large}% Text formatting
	{Part \arabic{section}}% Number
	{1em}% Space between number and text
	{}% Code before
	[\addvspace{-10pt}\rule{\textwidth}{0.4pt}]% Code after
\titleformat{\subsection}%
	{\color{blue}\normalsize\itshape}% Text formatting
	{Problem \arabic{section}.\arabic{subsection}}% Number
	{1em}% Space between number and text
	{}% Code before
	[\addvspace{-10pt}\rule{\widthof{Problem \arabic{section}.\arabic{subsection}}}{0.4pt}]% Code after
\titleformat{\subsubsection}%
	{\color{blue}}% Text formatting
	{\arabic{subsubsection} $\rightarrow$}% Number
	{1em}% Space between number and text
	{}% Code before
	[]% Code after

\definecolor{mygray}{rgb}{0.4, 0.4, 0.4}
\lstset{
style=Matlab-editor,
mlscaleinline=false,
basicstyle=\ttfamily\lst@ifdisplaystyle\scriptsize\fi,
frame=single,
rulecolor=\color{mygray},
numbers=left,
numbersep=10pt,
numberstyle=\footnotesize \ttfamily \color{mygray},
xleftmargin=30pt,
xrightmargin=5pt,
framexleftmargin=4pt,
framextopmargin=2pt
}

%%
%% COMMANDS
%%

% Draw legend lines for plots within the text. The \DeclareRobustCommand makes it work within figure captions.
\DeclareRobustCommand{\legendline}[1]{\raisebox{2pt}{\tikz{\draw[line width=2pt,#1](0,0) -- (5mm,0);}}}

% Superscript text: 1st, 2nd, 3rd, 4th
\newcommand{\suptext}[1]{\ensuremath{^\text{#1}}\xspace}
\newcommand{\st}{\suptext{st}}
\newcommand{\nd}{\suptext{nd}}
\newcommand{\rd}{\suptext{rd}}
\let\oldth\th % Reassign the current \th command
\renewcommand{\th}{\suptext{th}}

% Big O notation
\newcommand{\bigo}{\ensuremath{\mathcal{O}}}

% Fluid dynaimics terms
\newcommand{\ubar}{\ensuremath{\overline{u}}}
\newcommand{\Sbar}{\ensuremath{\overline{S}}}
\newcommand{\Wbar}{\ensuremath{\overline{W}}}

% Derivatives
\newcommand{\dd}[2]{\ensuremath{\frac{d #1}{d #2}}}
\newcommand{\pdd}[2]{\ensuremath{\frac{\partial #1}{\partial #2}}}
\newcommand{\tdd}[2]{\ensuremath{d #1 / d #2}}
\newcommand{\tpdd}[2]{\ensuremath{\partial #1 / \partial #2}}

% Bold vectors
\let\oldvec\vec
% Option 1: Works on more than single tokens, but makes regular letters italic as well as bold.
%\renewcommand{\vec}[1]{\mathbold{#1}}
% Option 2: Only works if a single token is passed to the command, but makes regular letters bold only.
\renewcommand{\vec}[1]{
	\ifcat\noexpand#1\relax
		\expandafter\mathbold
	\else
		\expandafter\mathbf
	\fi{{#1}}}

% Underlined vectors and double-underline matrices
\newcommand{\uvec}[1]{\ensuremath{\underline{#1}}}
\newcommand{\umat}[1]{\ensuremath{\underline{\underline{#1}}}}

% Engineering notation
\newcommand{\e}[1]{\ensuremath{\times 10^{#1}}}

% Expectation value and mean
\newcommand{\mean}[1]{\ensuremath{\overline{#1}}}
\newcommand{\expectation}[1]{\ensuremath{\left< #1 \right>}}
% Use the following inside text.
\newcommand{\texpectation}[1]{\ensuremath{\langle #1 \rangle}}

% Absolute value and norm bars
\DeclarePairedDelimiter\abs{\lvert}{\rvert}%
\DeclarePairedDelimiter\norm{\lVert}{\rVert}%
% Swap the definition of \abs* and \norm*, so that \abs
% and \norm resizes the size of the brackets, and the 
% starred version does not.
\makeatletter
\let\oldabs\abs
\def\abs{\@ifstar{\oldabs}{\oldabs*}}
\let\oldnorm\norm
\def\norm{\@ifstar{\oldnorm}{\oldnorm*}}
\makeatother

% Vertical asymptote for tikz/pgfplots
\newcommand{\vasymptote}[2][]{
    \draw [densely dashed,#1] ({rel axis cs:0,0} -| {axis cs:#2,0}) -- ({rel axis cs:0,1} -| {axis cs:#2,0});
}

% Vertical dirac delta function for tikz/pgfplots
\newcommand{\diracdelta}[2][]{
    \draw [#1] ({current axis.left of origin} -| {axis cs:#2,0}) -- ({rel axis cs:0,1} -| {axis cs:#2,0});
}

%%
%% DOCUMENT START
%%

\begin{document}

\pagestyle{fancyplain}
\lhead{}
\chead{}
\rhead{}
\lfoot{ASEN 6037: Project 2}
\cfoot{\thepage}
\rfoot{Ryan Skinner}

\noindent
{\Large \color{blue} Project 2}
\hfill
{\large Ryan Skinner}
\\[0.5ex]
{\large ASEN 6037: Turbulence}
\hfill
{\large Due 2015/05/08}
\hrule
\vspace{12pt}

In this project, we assess the strengths and shortcomings of Reynolds-averaged Navier-Stokes (RANS) models through derivations and comparison to DNS simulations of various flow classes.

\section{Derivation of RANS Models}

Please see the attached handwritten work; much of this problem consists of derivations which are long enough to preclude typesetting in a reasonable amount of time.

\section{Testing of RANS Models: Turbulent Channel Flow}

\subsection{}
%%
% Problem 2.1
%%

Consider a fully-developed turbulent channel flow. In such a flow, which of the components of $\ubar_i$, $\Sbar_{ij}$, and $\Wbar_{ij}$ are non-zero? What are $\tpdd{}{t}$ and $\ubar_i \tpdd{}{x_i}$? Comment on the validity of the equilibrium assumption used in Problem 1.8.

We will assume this turbulent channel flow to be either two- or three-dimensional, with the stream-wise direction denoted by $x$, and the span-wise direction(s) denoted by $y$.

Only $\ubar_x$ is non-zero, since this is the mean direction of flow. Any non-zero $\ubar_y$ would violate symmetry and, if near a wall, also the no-penetration boundary condition.

In the core flow away from the walls, all components of $\Sbar_{ij}$ are zero, since the velocity profile has negligible gradients in all directions ($\tpdd{}{x}$ is identically zero in a fully-developed flow). In the near-wall region, $\Sbar_{ij}$ does have non-zero components, because there is a large velocity gradient in the wall-normal direction. In this region, $\Sbar_{xy}$ is non-zero; it involves the derivative of the stream-wise velocity with respect to the wall-normal direction. All other components are zero because span-wise mean velocities are zero.

In a similar manner, all components of $\Wbar_{ij}$ are zero in the core flow. Again, this is because there exist no substantial velocity gradients in the core region. In the near-wall region, however, hairpin vortices peel off and create preferential vorticity. These structures create zero mean downstream and wall-normal vorticity due to symmetry of the hairpin, but non-zero vorticity in the direction of $(\text{wall-normal} \times \text{stream-wise})$. Taking the channel to be three-dimensional, if we consider the flow direction to be into the page $x$, along the bottom wall (whose normal is in the $y$-direction, pointing into the channel) vorticity will be preferentially in the $-\hat{z}$-direction. Thus, near this wall, $\Wbar_{xy}$ and $\Wbar_{yx}$ will be non-zero. Similar arguments apply to the other walls.

Since the flow is fully-developed, the averages of flow quantities will have no change in time, making $\tpdd{}{t}=0$. However, there will still be fluctuations in the flow for which $\tpdd{}{t} \neq 0$. For the convective derivative, we already know that spatial derivatives in the stream-wise direction must be zero, so $\ubar_x \tpdd{}{x}=0$. Additionally, there are no mean velocities in the span-wise directions, so $\ubar_y=0$. Thus all $\ubar_i \tpdd{}{x_i}=0$.

In conclusion, the equilibrium assumption that postulates constant anisotropy is an acceptable one for average quantities. Though this is not true for fluctuating, instantaneous quantities, we will presumably be using the RANS equations for modelling, in which only average quantities are used.

\subsection{}
%%
% Problem 2.2
%%

Using the turbulent channel flow DNS data from Moser, Kim, and Mansour (1999), we calculate and plot the non-zero components of $\ubar_i$, $\overline{u'_i u'_j}$, $k$, $\epsilon$, $a_{ij}$, and $\Sbar_{ij}$ as functions of both $y^+$ and $y/h$. Results are displayed in \figref{fig:part2_ubar} through \figref{fig:part2_Sij}

\begin{figure}[p]
\centering
\includegraphics[scale=0.85]{part2_ubar.png}
\vspace{6pt}
\caption{Mean velocity in the stream-wise direction.}
\label{fig:part2_ubar}
\end{figure}

\begin{figure}[p]
\centering
\includegraphics[scale=0.85]{part2_reystress.png}
\vspace{6pt}
\caption{Components of the Reynolds stress tensor.}
\label{fig:part2_reystress}
\end{figure}

\begin{figure}[p]
\centering
\includegraphics[scale=0.85]{part2_k.png}
\vspace{6pt}
\caption{Turbulence intensity.}
\label{fig:part2_k}
\end{figure}

\begin{figure}[p]
\centering
\includegraphics[scale=0.85]{part2_epsilon.png}
\vspace{6pt}
\caption{Turbulent kinetic energy dissipation.}
\label{fig:part2_epsilon}
\end{figure}

\begin{figure}[p]
\centering
\includegraphics[scale=0.85]{part2_aij.png}
\vspace{6pt}
\caption{Components of the anisotropy tensor.}
\label{fig:part2_aij}
\end{figure}

\begin{figure}[p]
\centering
\includegraphics[scale=0.85]{part2_Sij.png}
\vspace{6pt}
\caption{Components of the mean strain rate tensor; all components not shown are zero. Additionally, $\Sbar_{wv}$ should be zero in an ideal flow because there should be no mean span-wise velocities.}
\label{fig:part2_Sij}
\end{figure}

\subsection{}
%%
% Problem 2.3
%%

Using $a_{12}$ and $\Sbar_{12}$, the mean value of the eddy viscosity coefficient $C_\mu$ away from the walls ($y/h > 0.2$) is calculated to be approximately 0.086. This is the constant value that gives the closest agreement between the computational data and the closure model of
\begin{equation}
C_\mu =
\frac{-\epsilon a_{12}}{2k \Sbar_{12}}
.
\end{equation}
A plot of $C_\mu$ and its average value away from the wall is presented in \figref{fig:part2_Cmu}.

\begin{figure}[p]
\centering
\includegraphics[scale=0.85]{part2_Cmu.png}
\vspace{6pt}
\caption{Eddy viscosity coefficient $C_\mu$ as a function of distance from wall, calculated using $a_{12}$ and $\Sbar_{12}$. The average value is 0.086 for $y/h > 0.2$.}
\label{fig:part2_Cmu}
\end{figure}

\subsection{}
%%
% Problem 2.4
%%



\subsection{}
%%
% Problem 2.5
%%

\subsection{}
%%
% Problem 2.6
%%

\subsection{}
%%
% Problem 2.7
%%

\subsection{}
%%
% Problem 2.8
%%

\section{Testing of RANS Models: Unsteady Homogeneous Flow}

Problems 3.1--3.4 are derivation-heavy; they are not typeset here, but are included in the attached handwritten documents.
\setcounter{subsection}{4}

\subsection{}
%%
% Problem 3.5
%%

Assuming that $a_{12}=0$ at $t=0$, we numerically integrate the set of ordinary differential equations found in Problem 3.4 for the SKE and DKE models. We take $S^* \equiv S k_0 / \epsilon_0 = 3.3$, and examine the flow evolution for $(\omega/S) = \{0.01,0.1,0.5,1,10\}$. The evolution of $a_{12}$ is plotted as a function of $S \cdot t = S^* \tau$ for each of the $\omega/S$ values and both models. We furthermore assume that $C_\mu = 0.09$ in the SKE model.

\subsection{}
%%
% Problem 3.6
%%

\subsection{}
%%
% Problem 3.7
%%

\section{Testing of RANS Models: Computational Fluid Dynamics Code}

\subsection{}
%%
% Problem 4.1
%%

%%
%% DOCUMENT END
%%
\end{document}












